\documentclass[titlepage]{article}
\usepackage{verbatim}
\usepackage[T1]{fontenc}
\usepackage[utf8]{inputenc}
\usepackage[brazil]{babel}
\title{Proposta de Monografia}
\author{}

\begin{document}

\maketitle

\begin{description} 
\item[Alunos:]Cauê Haucke Porta Guerra\\Cecilia Fernandes\\Lucas Cavalcanti
\item[Supervisor: ] Prof. Dr. Alfredo Goldman
\end{description}

\subsection{Tema}
Aqui vai uma descrição inicial do tema

\subsection{Resumo}
Resumo da monografia a ser entregue

\subsection{Objetivos}
Lista com objetivos

\subsection{Atividades já realizadas}
Atividades ja realizadas. Pode ser pesquisa, desenvolvimento, etc

\subsection{Cronograma}
\begin{tabular}{|l|c|c|c|c|c|c|c|c|c|c|c|c|}
  \hline
  & Jan & Fev & Mar & Abr & Mai & Jun & Jul & Ago & Set & Out & Nov & Dez \\ \hline
	Tarefa1 & & X & & & & & & & & & & \\ \hline
  Tarefa2 & & & & X & X & & & & & & &\\
  \hline
\end{tabular}

\subsection{Estrutura esperada da parte técnica da monografia}

\subsubsection{Introdução}
Descrição do projeto e motivação.

\subsubsection{Conceitos e tecnologias estudadas}

\subsubsection{Implementação}
Descrição do que foi desenvolvido, e quais foram as escolhas feitas ao longo do projeto.

\subsubsection{Resultados obtidos}
Faz sentido?

\subsubsection{Conclusões}
Conclusões sobre o projeto, e análise do que foi feito ao longo do ano.

\subsubsection{Bibliografia}
Descrição da bibliografia utilizada na monografia. Segue a lista das fontes de pesquisa utilizadas até agora:

\end{document}
